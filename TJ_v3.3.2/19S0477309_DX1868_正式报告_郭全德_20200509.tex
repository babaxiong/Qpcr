
	\documentclass[UTF8]{ctexart}
		%-------------------------------------------------------------导言-------------------------
		\usepackage[T1]{fontenc} %设置text格式下划线
		\usepackage{lmodern} %设置text格式下划线
		\usepackage{geometry}
		\usepackage{multirow}
		\usepackage{float}%浮动包
		\usepackage{colortbl}%设置表格行高
		\definecolor{lightgray}{RGB}{245,245,245}
		\usepackage{makecell} %设置表格线
		\usepackage{booktabs} %表格线粗细
		\usepackage{fancyhdr}%插入页眉页脚页码包
		\usepackage{graphicx} % 图形宏包
		\usepackage{colortbl} %表格颜色包
		\usepackage{setspace}%使用间距宏包
		\usepackage{CJK,CJKnumb} %设置字体
		\usepackage{array}%表格固定列宽内容居中
		%\usepackage{natbib}
		%\usepackage[superscript]{cite} % 文献上标
		\usepackage[super,square,comma,sort&compress]{natbib}
		\usepackage[normalem]{ulem}%添加下划线
		\usepackage{lastpage}%获得总页数
		\usepackage{enumerate} %列表
		\usepackage{enumitem} %设置列表间隔
		\setlist[enumerate,1]{label=\arabic*、,leftmargin=7mm,labelsep=1.5mm,topsep=0mm,itemsep=-0.8mm}
		%添加水印
		\usepackage{tikz}
		\usepackage{xcolor}
		\usepackage{eso-pic}
		%添加水印
		\usepackage{longtable} %表格分页
		\usepackage{overpic} %封面添加患者信息

		%------------------------------------------------------结束-------------------------------
	%水印
	\newcommand\BackgroundPicture{
		 \put(0,0){
			\parbox[b][\paperheight]{\paperwidth}{
				\vfill
				      \centering
					  \begin{tikzpicture}[remember picture,overlay]
					  \node [rotate=60,scale=4,text opacity=0.5,font=\fontsize{10}{10}\selectfont] at (current page.center) {\textcolor{gray!80!cyan!30}{PMseq}};
					  \node [rotate=60,scale=4,text opacity=0.5,font=\fontsize{10}{10}\selectfont] at ([xshift=5cm,yshift=0cm]current page.west) {\textcolor{gray!80!cyan!30}{PMseq}};
					  \node [rotate=60,scale=4,text opacity=0.5,font=\fontsize{10}{10}\selectfont] at ([xshift=-5cm,yshift=0cm]current page.east) {\textcolor{gray!80!cyan!30}{PMseq}};

					  \node [rotate=60,scale=4,text opacity=0.5,font=\fontsize{10}{10}\selectfont] at ([xshift=0cm,yshift=-7cm]current page.north) {\textcolor{gray!80!cyan!30}{PMseq}};
					  \node [rotate=60,scale=4,text opacity=0.5,font=\fontsize{10}{10}\selectfont] at ([xshift=-5cm,yshift=-7cm]current page.north east) {\textcolor{gray!80!cyan!30}{PMseq}};
					  \node [rotate=60,scale=4,text opacity=0.5,font=\fontsize{10}{10}\selectfont] at ([xshift=5cm,yshift=-7cm]current page.north west) {\textcolor{gray!80!cyan!30}{PMseq}};

					  \node [rotate=60,scale=4,text opacity=0.5,font=\fontsize{10}{10}\selectfont] at ([xshift=0cm,yshift=7cm]current page.south) {\textcolor{gray!80!cyan!30}{PMseq}};
					  \node [rotate=60,scale=4,text opacity=0.5,font=\fontsize{10}{10}\selectfont] at ([xshift=5cm,yshift=7cm]current page.south west) {\textcolor{gray!80!cyan!30}{PMseq}};
					  \node [rotate=60,scale=4,text opacity=0.5,font=\fontsize{10}{10}\selectfont] at ([xshift=-5cm,yshift=7cm]current page.south east) {\textcolor{gray!80!cyan!30}{PMseq}};

			\end{tikzpicture}
			\vfill
			}
		}
	}
%水印
		\definecolor{mygray}{gray}{.9}
		%\newCJKfontfamilymsyh{微软雅黑}
		%\setCJKfamilyfont{yh}{Microsoft YaHei}
		\definecolor{myblue}{RGB}{0,74,143}


		\newcommand{\song}{\CJKfamily{song}}    % 宋体
		\newcommand{\fs}{\CJKfamily{fs}}             % 仿宋体
		\newcommand{\kai}{\CJKfamily{kai}}          % 楷体
		\newcommand{\hei}{\CJKfamily{hei}}         % 黑体
		\newcommand{\li}{\CJKfamily{li}}               % 隶书

		%-----------------------------------------------结束-------------------------
		\geometry{a4paper,centering,scale=0.8}
		\pagestyle{fancy} %插入页眉页脚
		%\graphicspath{{E:/工作/Plus产品小程序/Plus产品小程序版本更新/V3.3.2/TJ_v3.3.2/}}
		\graphicspath{{C:/CTEX/Pictures/}}

		%-------------------------------------------------页眉插入图片-------------
		\newsavebox{\headpic}
		\sbox{\headpic}{\includegraphics[height=2cm]{Y27-qPCR-7X-200309.jpg}} %设置页眉logo页眉
		\fancyhead[L]{\usebox{\headpic}}
		\fancyhead[C]{\zihao{-5}姓名:郭全德 \hspace{1cm}  采样日期:2020-05-07  \hspace{1cm}  样本编号:19S0477309}
		%--------------------------------------------------结束---------------------

		%----------------------------------------------设置页眉页脚格式------------
		
		\rhead{\zihao{-5} \uline{\hspace{14cm}DX-PMP-B92 V1.1} \vspace{0.1ex}   }
		 \lfoot{\zihao{-5} 客服电话:400-605-6655  \hspace{0.8cm} 网址:www.bgidx.cn}
		\cfoot{}%有该命令,页脚中间不出现页码
		%\rfoot{\thepage}
		\rfoot{\thepage \ / \pageref{LastPage}}
		\renewcommand{\headrulewidth}{0.2pt}%改为0pt即可去掉页脚上面的横线   
		\renewcommand{\footrulewidth}{0.2pt}
		%-------------------------------------------------结束--------------------
		\setlength{\extrarowheight}{4mm} %表格行高
		%\setlength{\parindent}{0pt}%段首不缩进
		%------------------------------------------开始正文--------------------------
		\begin{document}
		\bibliographystyle{unsrt} % 按文献在正文中引用的顺序排序
		\AddToShipoutPicture{\BackgroundPicture} %水印		
		%-----------------------------------------首页插入图片----------------------
		\newgeometry{left=-0.8cm,bottom=0cm,right=0.8cm,top=0cm}%更改单个页面页边距 
		\setcounter{page}{0} %页码1从第二页开始
		\thispagestyle{empty} %首页不显示页眉页脚

		%封面添加患者信息
		\begin{overpic}[width=\textwidth,height=\textheight,keepaspectratio]{F27-qPCR-7X-200309.png}
		\put(15,21){\begin{tabular}{cp{270 pt}<{\centering}}%19
		\textcolor{myblue}{\zihao{4}\bfseries \makebox[3.5em][s]{姓名}} & \textcolor{myblue}{\zihao{4}\bfseries 郭全德}\\
		\arrayrulecolor{myblue}\cline{2-2}
		\textcolor{myblue}{\zihao{4}\bfseries 样本编号} & \textcolor{myblue}{\zihao{4}\bfseries 19S0477309}\\
		\cline{2-2}
		\textcolor{myblue}{\zihao{4}\bfseries 样本类型} & \textcolor{myblue}{\zihao{4}\bfseries 肺泡灌洗液}\\
		\cline{2-2}
		\textcolor{myblue}{\zihao{4}\bfseries 检测项目} & \textcolor{myblue}{\zihao{4}\bfseries 七项呼吸道病原体核酸检测}\\
		\cline{2-2}
		\textcolor{myblue}{\zihao{4}\bfseries 送检单位} & \textcolor{myblue}{\zihao{4}\bfseries 泰安市中心医院}\\
		
		\cline{2-2}
		\end{tabular}}
		\end{overpic}		
		
		
		\restoregeometry %恢复到原来的页边距
		%------------------------------------------------------结束-----------------------------
		%\clearpage
		\newpage
		\topskip 1.5cm
		%\vspace{6mm}%页眉横线与正文间的垂直间距

		\noindent %顶格,不缩进
		{\hei\zihao{4}\bfseries 基本信息} %如何设置使得距离页眉的距离跟检测结果一样?
		%-------------------------------------------------------开始表格-----------------------------------------------------
		\begin{table}[H]
		%\small% 表格内容大小
		\zihao{-4}{\bfseries
		%\centering
		%\begin{tabular}{|lll|}
		\renewcommand\arraystretch{0.85} %设置表格行高
		\begin{tabular}{|p{0.33\textwidth}|p{0.33\textwidth}|p{0.33\textwidth}|} %设置表格宽度
		\hline
		\rowcolor{mygray}\multicolumn{3}{|c|}{受检者信息} \\%竖线的作用?
		\hline
		姓名:郭全德 & 性别:男 & 年龄: 39 \\
		\hline
		住院号:- & 床号:- & 原样本编号: -  \\
		\hline
		\rowcolor{mygray}\multicolumn{3}{|c|}{临床信息} \\
		\hline
		\multicolumn{3}{|p{\textwidth}|}{ 临床表现:-} \\
		\hline
		\multicolumn{3}{|p{\textwidth}|}{临床检测} \\
		\hline
		\end{tabular}
		\begin{tabular}{|p{0.243\textwidth}|p{0.24\textwidth}|p{0.24\textwidth}|p{0.24\textwidth}|} %设置表格宽度???????
		血WBC:- & 脑脊液WBC:- & 胸腹水WBC:- & CRP:- \\
		\hline
		PCT:- & 培养结果:- & 鉴定结果:- & 镜检结果:- \\
		\hline
		
		\multicolumn{4}{|p{\textwidth}|}{临床诊断:CT显示肺部感染} \\
		\hline
		\multicolumn{4}{|p{\textwidth}|}{重点关注病原:真菌,细菌,分枝杆菌,支原体/衣原体,寄生虫,病毒} \\
		\hline
		\multicolumn{4}{|p{\textwidth}|}{抗感染用药:-} \\
		\hline
		\rowcolor{mygray}\multicolumn{4}{|c|}{样本信息} \\
		\end{tabular}
		\begin{tabular}{|p{0.33\textwidth}|p{0.33\textwidth}|p{0.33\textwidth}|} %设置表格宽度
		\hline
		送检单位:泰安市中心医院 & 送检科室:呼吸内科 & 送检医师:杜医生  \\
		\hline
		采样日期:2020-05-07 & 收样日期:2020-05-08 & 报告日期:2020-05-09 11:20 \\
		\hline
		样本编号:19S0477309 & 样本类型:肺泡灌洗液 &  样本体积:-  \\
		%\Xhline{1pt}
	    \hline
		\end{tabular}
		\begin{tabular}{|p{0.33\textwidth}<{\centering}|p{0.33\textwidth}<{\centering}|p{0.33\textwidth}<{\centering}|} %设置表格宽度
		\rowcolor{mygray}\multicolumn{3}{|c|}{检测结果} \\
		\hline
		中文名 & 拉丁文名 & 检测结果\\	
		\hline	
		甲型流感病毒 & Influenza A virus,IAV &  \\
		\hline
		乙型流感病毒 & Influenza B virus,IBV &  \\
		\hline
		呼吸道合胞病毒 & Respiratory syncytial virus, RSV &  \\
		\hline
		人腺病毒 & Human adenovirus, HAdV & \\
		\hline
		人鼻病毒 & Human rhinovirus, HRV &  \\
		\hline
		肺炎支原体 & Mycoplasma Pneumoniae MP & \\
		\hline
		2019新型冠状病毒  & 2019-nCoV MP &  \\
		\hline
		\end{tabular}}
		\end{table}
		\newpage
		\topskip 0.1cm	
	\begin{longtable}{|p{1.04\textwidth}|} %设置表格宽度???????
		\hline
		\begin{spacing}{1}
		{\noindent\hei\bfseries\zihao{4} 致病性说明:}
		\end{spacing}
		
		\zihao{-4}\setlength{\baselineskip}{16pt}	\qquad甲型/乙型流感病毒(Influenza A/B virus,IAV/IBV)属于正粘病毒科(Orthomyxoviridae),为单链RNA病毒。这两种病毒均为常见的流感病毒,变异率高,流行率高,感染后的临床表现主要有发热、头痛、畏寒、乏力、恶心、咽痛、咳嗽和全身酸痛,严重病例可因肺炎、呼吸衰竭而致死亡$^{[1]}$。据世界卫生组织报道,流感病毒每年导致约300万-500万例流感病例,每年造成25万至50万人死亡,20万人住院。自1977年以来,甲型H1N1流感病毒(H1N1),甲型H3N2流感病毒(H3N2)和乙型流感病毒在全球共同传播$^{[2-3]}$。\\

		\zihao{-4}\setlength{\baselineskip}{16pt}	\qquad呼吸道合胞病毒(Respiratory syncytial virus,RSV)是一种RNA病毒,属于副粘病毒,该病毒经空气飞沫和密切接触传播,引起婴幼儿下呼吸道感染的主要病原;婴幼儿感染RSV后可发生严重的毛细支气管炎(简称毛支)和肺炎,与儿童哮喘有一定的关联,婴幼儿症状较重,可有高热、鼻炎、咽炎及喉炎,以后表现为细支气管炎及肺炎。少数病儿可并发中耳炎、胸膜炎及心肌炎等。成人和年长儿童感染后,主要表现为上呼吸道感染$^{[4]}$。\\

		\zihao{-4}\setlength{\baselineskip}{16pt}	\qquad人腺病毒(Human adenovirus, HAdV)为无包膜的双链DNA病毒,目前已发现至少90个基因型,分为A-G共7个亚属。呼吸道感染相关的HAdV主要有B亚属、C亚属和E亚属(HAdV-4型)。腺病毒肺炎约占社区获得性肺炎的4%-10%,重症肺炎以3型及7型多,是儿童社区获得性肺炎中较为严重的类型之一$^{[5]}$。\\

		\zihao{-4}\setlength{\baselineskip}{16pt}	\qquad人鼻病毒(Human rhinovirus,HRV)小RNA病毒科、肠病毒属的一种,是人患普通感冒的主要病原,对普通感冒尚无特异预防和治疗方法;有时会引起诸如哮喘、充血性心衰、支气管扩张,包囊纤维化等严重并发症,并且HRV多与其它呼吸道病毒合并感染,例如呼吸道合胞病毒、腺病毒等$^{[6]}$。\\

		\zihao{-4}\setlength{\baselineskip}{16pt}	\qquad肺炎支原体(M.Pneumonia,M.p)是一种大小介于细菌和病毒之间的致病微生物,支原体肺炎的病理改变以间质性肺炎为主,有时并发支气管肺炎,称为原发性非典型性肺炎。主要经飞沫传染,潜伏期2~3周,发病率以青少年最高。临床症状较轻,甚至根本无症状,若有也只是头痛、咽痛、发热、咳嗽等一般的呼吸道症状,但也有个别死亡报道。一年四季均可发生$^{[7]}$。\\
		
		\zihao{-4}\setlength{\baselineskip}{16pt}	\qquad2019新型冠状病毒(2019-nCoV)是2019年新发现的一种新型冠状病毒,属于β冠状病毒属,是2019新型冠状病毒疾病(COVID-19)的病原体,已在世界范围内广泛传播,并引起多个国家的COVID-19爆发。该病毒的传染性较强,潜伏期1-14天,无症状感染者也可能成为传染源,呼吸道飞沫传播及密切接触传播是主要的传播途径。该病毒常在COVID-19患者的呼吸道样本中发现,有文献报道在患者的粪便、尿液中也有检测到$^{[8-11]}$。\\
		\hline

		\begin{spacing}{1}
		{\noindent\hei\bfseries\zihao{4} 结论:}
		\end{spacing}
		\zihao{-4}
		\setlength{\baselineskip}{18pt}	\qquad本次检测中,呼吸道病原体检测结果为阳性,检出病原为:【任意病原名称,当出现多个需要用、隔开】。 \\  
		\setlength{\baselineskip}{18pt}	\qquad本次检测中,呼吸道病原体检测结果为阴性,未检出本产品检测范围内病原。\\
		\hline
		\end{longtable}
		
		\newpage
		\topskip 0.1cm
		\begin{longtable}{p{1.04\textwidth}}
		\begin{spacing}{1}
		{\noindent\hei\bfseries\zihao{4} 说明:}
		\end{spacing}
		\begin{enumerate}
		\item 本检测采用PCR扩增结合Sanger测序技术对肠道病毒进行分型检测。
		
		\item 由于肠道病毒亚型较多,序列存在突变或病毒载量较低、样本不合理采集等情况可能导致PCR 扩增结果为阴性。
		
		\item 若样本病毒拷贝数低于检出限,会显示检出肠道病毒样本,但Sanger测序失败,无法分型。
		
		\item 以上结论均为实验室检测数据,仅供临床参考,不能作为最终诊断结果。具体结果需结合临床体征、病史、其他实验室检查及治疗反应等情况综合考虑。
		
		\item 此报告结果仅对本次送检样本负责,报告相关解释须咨询临床医生。 
		
		\end{enumerate}
		\vspace{10ex} % 增加空行

		{\zihao{6}\color{white} \hspace{12cm}\$result\_seal\_user\_flag\_7\$}\\
		{\song\zihao{-4}\bfseries 检测者:}{\zihao{6}\color{white}\$result\_seal\_user\_flag\_15\$}
		{\song\zihao{-4}\bfseries 审核者:}{\zihao{6}\color{white}\$result\_seal\_user\_flag\_29\$}
		{\song\zihao{-4}\bfseries 报告日期:2020-05-09 11:20}
		
		\vspace{4cm}
		\begin{spacing}{1.5}
		{\noindent\song\bfseries\zihao{4} 附录}
		\end{spacing}
		\begin{spacing}{1.5}
		{\noindent\song\bfseries\zihao{-4} 参考文献}
		\end{spacing}

{\noindent\zihao{5}[1] Yamashita M, Krystal M, Fitch WM, Palese P. Influenza B virus evolution: co circulating lineages and comparison of evolutionary pattern with those of influenza A and C viruses. Virology[J]. 1988,163(1).112~122.}

{\noindent\zihao{5}[2] 舒跃龙等. 2004-2005年中国A(H1N1)亚型流感病毒抗原性及基因特性研究[J].临床医学, 2006,20(2):27~29.}

{\noindent\zihao{5}[3] 陈继明, 郭元吉. 乙型流行性感冒病毒两大谱系的起源及其演变特征[J]. 病毒学报, 2001,17(4):322~327.}

{\noindent\zihao{5}[4] 林立, 李昌崇. 呼吸道合胞病毒感染发病机制[J]. 中华儿科杂志 2006,44(9):673~675.}

{\noindent\zihao{5}[5] 高文娟, 金玉, 段招军. 人腺病毒的研究进展[J]. 病毒学报, 2014,30(2):193~200.}

{\noindent\zihao{5}[6] 王焕焕, 毛乃颖, 王善振等. 人鼻病毒的研究进展[J]. 病毒学报, 2011,27(3):294~297.}

{\noindent\zihao{5}[7] 陆权, 陆敏. 肺炎支原体感染的流行病学[J]. 实用儿科临床杂志, 2007,22(4):241~243.}

{\noindent\zihao{5}[8] Lu R, Zhao X, Li J et al.. Genomic characterisation and epidemiology of 2019 novel coronavirus: implications for virus origins and receptor binding. Lancet. 2020 Feb 22;395(10224):565-574.}

{\noindent\zihao{5}[9] Zhu N, Zhang D, Wang W, et al.. A Novel Coronavirus from Patients with Pneumonia in China, 2019. N Engl J Med. 2020 Feb 20;382(8):727-733.}

{\noindent\zihao{5}[10] Xie C, Jiang L, Huang G, et al.. Comparison of different samples for 2019 novel coronavirus detection by nucleic acid amplification tests. Int J Infect Dis. 2020 Feb 27;93:264-267.}

{\noindent\zihao{5}[11] Ling Y, Xu SB, Lin YX, et al.. Persistence and clearance of viral RNA in 2019 novel coronavirus disease rehabilitation patients. Chin Med J (Engl). 2020 Feb 28. doi: 10.1097/CM9.0000000000000774. [Epub ahead of print]}

\end{longtable}
\end{document}
