
	\documentclass[UTF8]{ctexart}
		%-------------------------------------------------------------导言-------------------------
		\usepackage[T1]{fontenc} %设置text格式下划线
		\usepackage{lmodern} %设置text格式下划线
		\usepackage{geometry}
		\usepackage{multirow}
		\usepackage{float}%浮动包
		\usepackage{colortbl}%设置表格行高
		\definecolor{lightgray}{RGB}{245,245,245}
		\usepackage{makecell} %设置表格线
		\usepackage{booktabs} %表格线粗细
		\usepackage{fancyhdr}%插入页眉页脚页码包
		\usepackage{graphicx} % 图形宏包
		\usepackage{colortbl} %表格颜色包
		\usepackage{setspace}%使用间距宏包
		\usepackage{CJK,CJKnumb} %设置字体
		\usepackage{array}%表格固定列宽内容居中
		%\usepackage{natbib}
		%\usepackage[superscript]{cite} % 文献上标
		\usepackage[super,square,comma,sort&compress]{natbib}
		\usepackage[normalem]{ulem}%添加下划线
		\usepackage{lastpage}%获得总页数
		\usepackage{enumerate} %列表
		\usepackage{enumitem} %设置列表间隔
		\setlist[enumerate,1]{label=\arabic*、,leftmargin=7mm,labelsep=1.5mm,topsep=0mm,itemsep=-0.8mm}
		%添加水印
		\usepackage{tikz}
		\usepackage{xcolor}
		\usepackage{eso-pic}
		%添加水印
		\usepackage{longtable} %表格分页
		\usepackage{overpic} %封面添加患者信息

		%------------------------------------------------------结束-------------------------------
	%水印
	\newcommand\BackgroundPicture{
		 \put(0,0){
			\parbox[b][\paperheight]{\paperwidth}{
				\vfill
				      \centering
					  \begin{tikzpicture}[remember picture,overlay]
					  \node [rotate=60,scale=4,text opacity=0.5,font=\fontsize{10}{10}\selectfont] at (current page.center) {\textcolor{gray!80!cyan!30}{PMseq}};
					  \node [rotate=60,scale=4,text opacity=0.5,font=\fontsize{10}{10}\selectfont] at ([xshift=5cm,yshift=0cm]current page.west) {\textcolor{gray!80!cyan!30}{PMseq}};
					  \node [rotate=60,scale=4,text opacity=0.5,font=\fontsize{10}{10}\selectfont] at ([xshift=-5cm,yshift=0cm]current page.east) {\textcolor{gray!80!cyan!30}{PMseq}};

					  \node [rotate=60,scale=4,text opacity=0.5,font=\fontsize{10}{10}\selectfont] at ([xshift=0cm,yshift=-7cm]current page.north) {\textcolor{gray!80!cyan!30}{PMseq}};
					  \node [rotate=60,scale=4,text opacity=0.5,font=\fontsize{10}{10}\selectfont] at ([xshift=-5cm,yshift=-7cm]current page.north east) {\textcolor{gray!80!cyan!30}{PMseq}};
					  \node [rotate=60,scale=4,text opacity=0.5,font=\fontsize{10}{10}\selectfont] at ([xshift=5cm,yshift=-7cm]current page.north west) {\textcolor{gray!80!cyan!30}{PMseq}};

					  \node [rotate=60,scale=4,text opacity=0.5,font=\fontsize{10}{10}\selectfont] at ([xshift=0cm,yshift=7cm]current page.south) {\textcolor{gray!80!cyan!30}{PMseq}};
					  \node [rotate=60,scale=4,text opacity=0.5,font=\fontsize{10}{10}\selectfont] at ([xshift=5cm,yshift=7cm]current page.south west) {\textcolor{gray!80!cyan!30}{PMseq}};
					  \node [rotate=60,scale=4,text opacity=0.5,font=\fontsize{10}{10}\selectfont] at ([xshift=-5cm,yshift=7cm]current page.south east) {\textcolor{gray!80!cyan!30}{PMseq}};

			\end{tikzpicture}
			\vfill
			}
		}
	}
%水印
		\definecolor{mygray}{gray}{.9}
		%\newCJKfontfamilymsyh{微软雅黑}
		%\setCJKfamilyfont{yh}{Microsoft YaHei}
		\definecolor{myblue}{RGB}{0,74,143}


		\newcommand{\song}{\CJKfamily{song}}    % 宋体
		\newcommand{\fs}{\CJKfamily{fs}}             % 仿宋体
		\newcommand{\kai}{\CJKfamily{kai}}          % 楷体
		\newcommand{\hei}{\CJKfamily{hei}}         % 黑体
		\newcommand{\li}{\CJKfamily{li}}               % 隶书

		%-----------------------------------------------结束-------------------------
		\geometry{a4paper,centering,scale=0.8}
		\pagestyle{fancy} %插入页眉页脚
		%\graphicspath{{E:/工作/Plus产品小程序/Plus产品小程序版本更新/V3.3.2/TJ_v3.3.2/}}
		\graphicspath{{C:/CTEX/Pictures/}}

		%-------------------------------------------------页眉插入图片-------------
		\newsavebox{\headpic}
		\sbox{\headpic}{\includegraphics[height=2cm]{.jpg}} %设置页眉logo页眉
		\fancyhead[L]{\usebox{\headpic}}
		\fancyhead[C]{\zihao{-5}姓名:郭全德 \hspace{1cm}  采样日期:2020-05-07  \hspace{1cm}  样本编号:19S0477309}
		%--------------------------------------------------结束---------------------

		%----------------------------------------------设置页眉页脚格式------------
		
		\rhead{\zihao{-5} \uline{\hspace{14cm}DX-PMP-B V1.1} \vspace{0.1ex}   }
		 \lfoot{\zihao{-5} 客服电话:400-605-6655  \hspace{0.8cm} 网址:www.bgidx.cn}
		\cfoot{}%有该命令,页脚中间不出现页码
		%\rfoot{\thepage}
		\rfoot{\thepage \ / \pageref{LastPage}}
		\renewcommand{\headrulewidth}{0.2pt}%改为0pt即可去掉页脚上面的横线   
		\renewcommand{\footrulewidth}{0.2pt}
		%-------------------------------------------------结束--------------------
		\setlength{\extrarowheight}{4mm} %表格行高
		%\setlength{\parindent}{0pt}%段首不缩进
		%------------------------------------------开始正文--------------------------
		\begin{document}
		\bibliographystyle{unsrt} % 按文献在正文中引用的顺序排序
		\AddToShipoutPicture{\BackgroundPicture} %水印		
		%-----------------------------------------首页插入图片----------------------
		\newgeometry{left=-0.8cm,bottom=0cm,right=0.8cm,top=0cm}%更改单个页面页边距 
		\setcounter{page}{0} %页码1从第二页开始
		\thispagestyle{empty} %首页不显示页眉页脚

		%封面添加患者信息
		\begin{overpic}[width=\textwidth,height=\textheight,keepaspectratio]{.png}
		\put(15,21){\begin{tabular}{cp{270 pt}<{\centering}}%19
		\textcolor{myblue}{\zihao{4}\bfseries \makebox[3.5em][s]{姓名}} & \textcolor{myblue}{\zihao{4}\bfseries 郭全德}\\
		\arrayrulecolor{myblue}\cline{2-2}
		\textcolor{myblue}{\zihao{4}\bfseries 样本编号} & \textcolor{myblue}{\zihao{4}\bfseries 19S0477309}\\
		\cline{2-2}
		\textcolor{myblue}{\zihao{4}\bfseries 样本类型} & \textcolor{myblue}{\zihao{4}\bfseries 肺泡灌洗液}\\
		\cline{2-2}
		\textcolor{myblue}{\zihao{4}\bfseries 检测项目} & \textcolor{myblue}{\zihao{4}\bfseries }\\
		\cline{2-2}
		\textcolor{myblue}{\zihao{4}\bfseries 送检单位} & \textcolor{myblue}{\zihao{4}\bfseries 泰安市中心医院}\\
		
		\cline{2-2}
		\end{tabular}}
		\end{overpic}		
		
		
		\restoregeometry %恢复到原来的页边距
		%------------------------------------------------------结束-----------------------------
		%\clearpage
		\newpage
		\topskip 1.5cm
		%\vspace{6mm}%页眉横线与正文间的垂直间距

		\noindent %顶格,不缩进
		{\hei\zihao{4}\bfseries 基本信息} %如何设置使得距离页眉的距离跟检测结果一样?
		%-------------------------------------------------------开始表格-----------------------------------------------------
		\begin{table}[H]
		%\small% 表格内容大小
		\zihao{-4}{\bfseries
		%\centering
		%\begin{tabular}{|lll|}
		\renewcommand\arraystretch{0.85} %设置表格行高
		\begin{tabular}{|p{0.33\textwidth}|p{0.33\textwidth}|p{0.33\textwidth}|} %设置表格宽度
		\hline
		\rowcolor{mygray}\multicolumn{3}{|c|}{受检者信息} \\%竖线的作用?
		\hline
		姓名:郭全德 & 性别:男 & 年龄: 39 \\
		\hline
		住院号:- & 床号:- & 原样本编号: -  \\
		\hline
		\rowcolor{mygray}\multicolumn{3}{|c|}{临床信息} \\
		\hline
		\multicolumn{3}{|p{\textwidth}|}{ 临床表现:-} \\
		\hline
		\multicolumn{3}{|p{\textwidth}|}{临床检测} \\
		\hline
		\end{tabular}
		\begin{tabular}{|p{0.243\textwidth}|p{0.24\textwidth}|p{0.24\textwidth}|p{0.24\textwidth}|} %设置表格宽度???????
		血WBC:- & 脑脊液WBC:- & 胸腹水WBC:- & CRP:- \\
		\hline
		PCT:- & 培养结果:- & 鉴定结果:- & 镜检结果:- \\
		\hline
		
		\multicolumn{4}{|p{\textwidth}|}{临床诊断:CT显示肺部感染} \\
		\hline
		\multicolumn{4}{|p{\textwidth}|}{重点关注病原:真菌,细菌,分枝杆菌,支原体/衣原体,寄生虫,病毒} \\
		\hline
		\multicolumn{4}{|p{\textwidth}|}{抗感染用药:-} \\
		\hline
		\rowcolor{mygray}\multicolumn{4}{|c|}{样本信息} \\
		\end{tabular}
		\begin{tabular}{|p{0.33\textwidth}|p{0.33\textwidth}|p{0.33\textwidth}|} %设置表格宽度
		\hline
		送检单位:泰安市中心医院 & 送检科室:呼吸内科 & 送检医师:杜医生  \\
		\hline
		采样日期:2020-05-07 & 收样日期:2020-05-08 & 报告日期:2020-05-09 11:20 \\
		\hline
		样本编号:19S0477309 & 样本类型:肺泡灌洗液 &  样本体积:-  \\
		%\Xhline{1pt}
	    \hline
		\end{tabular}
		